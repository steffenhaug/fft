\PassOptionsToPackage{dvipsnames}{xcolor}
\documentclass[14pt]{beamer}

\usepackage{amsmath}
\usepackage{mathtools}
\usepackage{emoji}
\usepackage{xfrac}
\usepackage{cancel}
\usepackage{csquotes}

\usepackage[]{xcolor}
\def\MyGreen{ForestGreen}
\def\MyRed{Red}
\def\MyOrange{BurntOrange}
\def\MyBlue{NavyBlue}
\definecolor{SuperLightGray}{rgb}{0.9,0.9,0.9}

\usepackage{calc}

% This lets us \includestandalone tikz pictures in standalone documents.
\usepackage{tikz}
\usetikzlibrary{tikzmark}
\usepackage[mode=buildnew]{standalone}
\usepackage{xifthen}

\makeatother
\renewcommand{\thefootnote}{
    \ifcase\value{footnote}
        \or*
        \or**
        \or***
        \or****
    \fi}
\makeatletter

\usefonttheme{serif}
\setbeamertemplate{navigation symbols}{}%remove navigation symbols

\setbeamerfont{title}{shape=\sc}
\setbeamerfont{frametitle}{shape=\sc}

\setbeamercolor{title}{fg=black}
\setbeamercolor{frametitle}{fg=black}

\newcommand{\Seq}[1]{\left\{ {#1} \right\}}
\newcommand{\Exp}[1]{e^{#1}}

\title{The Fourier transform and the FFT algorithm}
\author{Steffen Haug}
\date{}

\begin{document}
\begin{frame}
\titlepage
\end{frame}

\begin{frame}
    \centering
    \textsc{The Fourier transform}\\[1em]
    \begin{equation}
        \hat f (\xi) = \int\displaylimits_{\mathbb R} f(t) \Exp{-2 \pi i \xi t } dt
    \end{equation}\\[1em]
    % The fourier transform: Incredibly important for engineering applications
    % - The fourier transform f_hat gives for a frequency xi the "amount" of xi in the signal
    % - Unfeasible to evaluate automatically analytically

    \pause
    Analytic evaluation generally not feasible.
\end{frame}

\begin{frame}
    \begin{equation*}
        \hat f (\xi) = \int\displaylimits_{\mathbb R} f(t) \Exp{-2 \pi i \xi t } dt
    \end{equation*}\\[1em]

    Most engineering applications use its discrete counterpart -- the DFT:
    \begin{equation}
    X_k = \sum_{n=0}^{N-1} x_n \Exp{-\frac {2 \pi i k n} N}
    \label{eq:dft}
    \end{equation}
    % The sequence X is a sort of discrete counterpart to f_hat
    % Imagine x samples f
\end{frame}

\begin{frame}
    \begin{equation*}
    X_k = \sum_{n=0}^{N-1} x_n \Exp{-\frac {2 \pi i k n} N}
    \label{eq:dft}
    \end{equation*}\\[1em]
    \centering
    We want a way to compute this \textit{efficiently}.
\end{frame}



\begin{frame}
    \centering
    \textsc{\large Preliminaries}\\
    \textit{\scriptsize Eulers formula and the roots of unity}
\end{frame}


% eulers identiry: special case of eulers formula exp(ix) = cos = isin

\begin{frame}
    \centering
    \textsc{Eulers identity}\\[1em]

    \begin{equation}
        e^{i \pi} = -1
    \end{equation}
\end{frame}

\begin{frame}
    \begin{equation*}
        e^{i \pi} = -1
    \end{equation*}\\[1em]

    \centering
    Just a special case of $e^{i \vartheta}$ when $\vartheta = \pi$.
    \begin{equation}
        e^{i\vartheta} = \cos{\vartheta} + i \sin{\vartheta}
    \end{equation}
\end{frame}


\begin{frame}
    \begin{equation*}
        e^{i\vartheta} = \cos{\vartheta} + i \sin{\vartheta}
    \end{equation*}
    \begin{figure}
        \centering
        \includestandalone[width=.625\textwidth]{euler}
    \end{figure}
\end{frame}

\begin{frame}
    \begin{equation*}
        e^{i\pi} = -1
    \end{equation*}
    \begin{figure}
        \centering
        \includestandalone[width=.625\textwidth]{euler_pi}
    \end{figure}
\end{frame}

% roots of unity: exp(in/N) solutions to x^N = 1
% remember complex multiplication: multiply length (=1) and add angles. Arg xy = Arg x + Arg y
% symmetry: N even => each root has a opposite partner with twice the angle which just has a negative sign

\begin{frame}
    \centering
    \textsc{The roots of unity}\\[1em]
    \begin{equation*}
        z^N = 1
    \end{equation*}
\end{frame}

\begin{frame}
    \begin{equation*}
        z^N = 1
    \end{equation*}

    \begin{figure}
        \centering
        \includestandalone[width=.5\textwidth]{roots}
    \end{figure}
    \centering

    \textsc{Solutions}: $\left(\!e^\frac {2\pi i}{N}\right)^{\!\!n}$ for $n$ in $0, \dots, N-1$.
\end{frame}

% \begin{frame}
%     \centering
%     \textsc{To verify}\\[1em]
%     \begin{align*}
%         \left(e^\frac{2 \pi i n} N\right)^N &=
%             \left(e^\frac{2 \pi i N} N\right)^n \\
%             &= \left(e^{2 \pi i}\right)^n\\
%             &= 1^n \\
%             &= 1
%     \end{align*}
% \end{frame}

% \begin{frame}
%     \centering
%     \textsc{Important point 1}
%     \begin{figure}
%         \centering
%         \includestandalone[width=.5\textwidth]{roots_mirror}
%     \end{figure}
%     When N is even, every root has a ``friend'' with the opposite sign.
% \end{frame}

% \begin{frame}
%     \centering
%     \textsc{Important point 2}
%     \begin{figure}
%         \centering
%         \raisebox{-.5\height}{ \includestandalone[width=.3\textwidth]{roots_8} }
%         $\overset{z \;\longmapsto\; z^2}{\xrightarrow{\hspace*{2cm}}}$
%         \raisebox{-.5\height}{ \includestandalone[width=.3\textwidth]{roots_4} }
%     \end{figure}
%     Squaring the roots of unity, gives the roots of unity of half the order.
% \end{frame}

\begin{frame}
    Let $W = e^\frac{-2 \pi i} N$ (notation from Cooley-Tukey).

    \begin{figure}
        \centering
        \includestandalone[width=.4\textwidth]{roots_dft}
    \end{figure}

    We can restate the DFT \eqref{eq:dft} as
    \begin{equation}
        X_k = \sum_{n=0}^{N-1} x_n W^{nk}
        \label{eq:dftw}
    \end{equation}
\end{frame}

\begin{frame}
    \begin{center}
        \textsc{Foreshadowing}\\[.5em]
        \scriptsize{\textsc{The goal} is to recursively decompse the DFT in such a way that {\em intermediate calculations}
        are {\em overlapping} or {\em symmetrically opposite} on the unit circle.}\\[1em]

        \tiny{Thus, they may be reused directly or by simply a change of sign.}
    \end{center}
\end{frame}

\begin{frame}
    \begin{equation*}
        X_k = \sum_{n=0}^{N-1} x_n W^{nk}
    \end{equation*}

    \centering
    So what is the DFT \textit{actually} doing?\\[1em]

    \pause

    \begin{figure}
        \centering
        \includestandalone[width=.4\textwidth]{roots_dft_k1}
        \includestandalone[width=.4\textwidth]{roots_dft_k3}
    \end{figure}

    \scriptsize{Just multiplying our N array values by the Nth roots of unity!}
\end{frame}

\begin{frame}
    \begin{equation*}
        X_k = \sum_{n=0}^{N-1} x_n W^{nk}
    \end{equation*}

    \centering
    So what is the DFT \textit{actually} doing?\\[1em]

    \begin{figure}
        \centering
        \includestandalone[width=.4\textwidth]{roots_dft_k1}
        \includestandalone[width=.4\textwidth]{roots_dft_k3}
    \end{figure}

    \scriptsize{$k$ is the ``stride'' along the roots of unity -- the ``frequency''.}

    \tiny{(If $k$ is larger, we go faster around the circle)}
\end{frame}

\begin{frame}
    \centering
    \textsc{\large The key idea}\\
    \textit{\scriptsize Decomposing the DFT}
\end{frame}


\begin{frame}
    \centering
    We can decompose \eqref{eq:dftw} into a a sum of sums:
    \begin{align*}
        X_k &= \sum_{n=0}^{N-1} x_n W^{nk} \\
            &= \underbrace{\sum_{n=0}^{\mathclap{N/2-1}} x_{2n}    W^{(2n)  k}}_{\text{Even indices}}            
             + \underbrace{\sum_{n=0}^{\mathclap{N/2-1}} x_{2n+1}  W^{(2n+1)k}}_{\text{Odd indices}}
    \end{align*}
    \tiny{We can just as well decompose it in three sums that sum every 3rd value and so on.}\\[1em]
\end{frame}

\begin{frame}
    \centering
    \textsc{Remember}\\[1em]
    \small{An N-point DFT uses the Nth roots of unity!}
\end{frame}

\begin{frame}
    \begin{figure}
        \centering
        \begin{minipage}{.4\textwidth}
            \centering
            \includestandalone[width=\textwidth]{roots_dft_even}
            $$\sum_{n=0}^{\mathclap{N/2-1}} x_{2n}{\color{\MyGreen}W}^{{\color{\MyGreen}2n}k}$$
        \end{minipage}
        \begin{minipage}{.4\textwidth}
            \centering
            \includestandalone[width=\textwidth]{roots_dft_odd}
            $$\sum_{n=0}^{\mathclap{N/2-1}} x_{2n+1}{\color{\MyRed}W}^{{\color{\MyRed}(2n+1)}k}$$
        \end{minipage}
    \end{figure}
    \centering
    \small{The odd-index sum does not use  the N/2th roots of unity.}
\end{frame}

\begin{frame}
    \begin{figure}
        \centering
        \begin{minipage}{.4\textwidth}
            \centering
            \includestandalone[width=\textwidth]{roots_dft_even}
            $$\sum_{n=0}^{\mathclap{N/2-1}} x_{2n}{\color{\MyGreen}W}^{{\color{\MyGreen}2n}k}$$
        \end{minipage}
        \begin{minipage}{.4\textwidth}
            \centering
            \includestandalone[width=\textwidth]{roots_dft_odd}
            $$\sum_{n=0}^{\mathclap{N/2-1}} x_{2n+1}{\color{\MyRed}W}^{{\color{\MyRed}(2n+1)}k}$$
        \end{minipage}
    \end{figure}
    \centering
    \small{It is clearly not a proper N/2-point DFT!}
\end{frame}

\begin{frame}
    \begin{figure}
        \centering
        \begin{minipage}{.4\textwidth}
            \centering
            \includestandalone[width=\textwidth]{roots_dft_even}
            $$\sum_{n=0}^{\mathclap{N/2-1}} x_{2n}{\color{\MyGreen}W}^{{\color{\MyGreen}2n}k}$$
        \end{minipage}
        \begin{minipage}{.4\textwidth}
            \centering
            \includestandalone[width=\textwidth]{roots_dft_odd_fix}
            $${\color{\MyRed}W^k}\sum_{n=0}^{
                \mathclap{N/2-1}} x_{2n+1}{\color{\MyGreen}W}^{{\color{\MyGreen}2n}k}$$
        \end{minipage}
    \end{figure}
    \centering
    \small{Factor out $W^k$ to get what seems like a proper DFT.}
\end{frame}

\begin{frame}
    \centering
    \textsc{Correct decomposition of the DFT}
    \begin{alignat}{3}
        X_k &=                       {\color{\MyBlue}  \sum_{n=0}^{\mathclap{N/2-1}} x_{2n}   \tikzmark{a}W^{2nk}}
                &&+ \tikzmark{b} W^k {\color{\MyOrange}\sum_{n=0}^{\mathclap{N/2-1}} x_{2n+1} \tikzmark{c}W^{2nk}}
                \label{eq:decomposed}
    \end{alignat}\\[1em]

    \onslide<2->{
        % putting the picture after \pause breaks it for some reason
        % but this accomplishes the same thing
    \begin{tikzpicture}[remember picture,overlay]
        \draw[<-] ([shift={(.5em,-4pt)}]pic cs:a) -- ([shift={(0pt,-20pt)}]pic cs:a) 
            node[anchor=north] {\small \bf \color{\MyRed} !}; 
        \draw[<-] ([shift={(.5em,-4pt)}]pic cs:b) -- ([shift={(0pt,-20pt)}]pic cs:b) 
            node[anchor=north] {\small \bf \color{\MyRed} !}; 
        \draw[<-] ([shift={(.5em,-4pt)}]pic cs:c) -- ([shift={(0pt,-20pt)}]pic cs:c) 
            node[anchor=north] {\small \bf \color{\MyRed} !}; 
    \end{tikzpicture}
    }

    \pause




    But wait...\\[1em]

    \small{${\color{\MyBlue}E_k}, {\color{\MyOrange}O_k}$ is supposedly N/2-point DFTs,
        but $0 \leq k < N$?
    }

\end{frame}

\begin{frame}
    \centering
    \textsc{$\overset{\mathclap{\text{Almost correct}}}{\cancel{\text{Correct}}}$
        decomposition of the DFT}
    \begin{alignat}{3}
        X_k &=                       {\color{\MyBlue}  \sum_{n=0}^{\mathclap{N/2-1}} x_{2n}   \tikzmark{d}W^{2nk}}
                &&+ \tikzmark{e} W^k {\color{\MyOrange}\sum_{n=0}^{\mathclap{N/2-1}} x_{2n+1} \tikzmark{f}W^{2nk}}
                \tag{\ref{eq:decomposed}}
    \end{alignat}\\[1em]

    \begin{tikzpicture}[remember picture,overlay]
        \draw[<-] ([shift={(.5em,-4pt)}]pic cs:d) -- ([shift={(0pt,-20pt)}]pic cs:d) 
            node[anchor=north] {\small \bf \color{\MyRed} !}; 
        \draw[<-] ([shift={(.5em,-4pt)}]pic cs:e) -- ([shift={(0pt,-20pt)}]pic cs:e) 
            node[anchor=north] {\small \bf \color{\MyRed} !}; 
        \draw[<-] ([shift={(.5em,-4pt)}]pic cs:f) -- ([shift={(0pt,-20pt)}]pic cs:f) 
            node[anchor=north] {\small \bf \color{\MyRed} !}; 
    \end{tikzpicture}

    \Huge \emoji{face-with-raised-eyebrow}
\end{frame}

\begin{frame}
    \centering
    \tiny{\textsc{In fact}, the sums yield the correct result if evaluated directly for every $k$,
    but fails to take advantage of any symmetry, and thus gives no performance benefit.}
\end{frame}

\begin{frame}
    \centering
    \textsc{\large Gaining speed}\\
    \textit{\scriptsize Reusing intermediate results}
\end{frame}

\begin{frame}
    \centering
    Let's look closer at $W^{2nk}$\\[.5em]
    \begin{figure}
        \centering
        \includestandalone[width=.6\textwidth]{roots_dft_w2nk_symm}
    \end{figure}

    \tiny{Upper half range of $W^{2nk}$ overlaps with lower half for $N=1$}
\end{frame}

\begin{frame}
    \centering
    \begin{figure}
        \centering
        \includestandalone[width=.6\textwidth]{roots_dft_w2nk_symm}
    \end{figure}
    {\tiny This holds in general:}
    $$
    W^{2n(k + N/2)}=W^{2nk} \underbrace{W^\frac{\cancel 2nN}{\cancel 2}}_{=1}=W^{2nk}
    $$
\end{frame}

\begin{frame}
    \centering
    Now, let's look closer at $W^{k}$\\[.5em]
    \begin{figure}
        \centering
        \includestandalone[width=.5\textwidth]{roots_dft_wnhalf}
    \end{figure}
    $$
    W^{k+N/2} = W^k{\color{\MyBlue} W^{N/2}} = -W^k
    $$

    \tiny{Upper range of $W^k$ symmetrically opposite to lower range!}
\end{frame}


\begin{frame}
    \begin{alignat}{3}
        X_k &=                       {\color{\MyBlue}  \sum_{n=0}^{\mathclap{N/2-1}} x_{2n}   \tikzmark{g}W^{2nk}}
                &&+ \tikzmark{h} W^k {\color{\MyOrange}\sum_{n=0}^{\mathclap{N/2-1}} x_{2n+1} \tikzmark{i}W^{2nk}}
                \tag{\ref{eq:decomposed}}
    \end{alignat}\\[1em]

    \begin{tikzpicture}[remember picture,overlay]
        \draw[<-] ([shift={(.5em,-.2em)}]pic cs:g) -- ([shift={(.0em,-1.5em)}]pic cs:g) 
            node[anchor=north] {\tiny Overlapping}; 
        \draw[<-] ([shift={(.5em,-.2em)}]pic cs:h) -- ([shift={(-.442em,-2.5em)}]pic cs:h) 
            node[anchor=north] {\tiny Symmetrically opposite}; 
        \draw[<-] ([shift={(.5em,-.2em)}]pic cs:i) -- ([shift={(.0em,-1.5em)}]pic cs:i) 
            node[anchor=north] {\tiny Overlapping}; 
    \end{tikzpicture}

    \pause

    \centering
    \textsc{Decomposition}
    \begin{equation}
        \left\{
            \begin{array}{lcl}
                X_k       &=& {\color{\MyBlue}E_k} + W^k{\color{\MyOrange}O_k} \\
                X_{k+N/2} &=& {\color{\MyBlue}E_{k}} - W^{k}{\color{\MyOrange}O_{k}}
            \end{array}
        \right.
    \end{equation}
    \scriptsize $k \in [0, N/2 - 1]$
\end{frame}

\begin{frame}
    \centering
    \textsc{Algorithm}
    \begin{figure}
        \centering
        \includestandalone[width=.8\textwidth]{butterfly}
    \end{figure}
    \scriptsize{
        \only<1>{$\Seq{x_n}$ ordered by index parity.}
        \only<2>{Recursively compute the N/2-point FFTs $E$ and $O$.}
        \only<3>{$X_0 = E_0 + W^0\!O_0$...}
        \only<4>{...and so on for all $k < N/2$.}
        \only<5>{$X_5 = E_0 - W^0\!O_0$...}
        \only<6>{...and so on for all $k \geq N/2.$}
    }
\end{frame}

\begin{frame}
    \centering
    \textsc{Complexity analysis}

    {\tiny Just for the record}

    $$
    T(n) = \left\{
        \begin{array}{ll}
            2T(n/2) + n & \text{if } n > 1 \\
            1           & \text{if } n = 1
        \end{array}
    \right.
    $$
    {\tiny This is not very deep -- straight forward application of the \textsc{Master Theorem}:}
    $$
    \log_2 2 = 1 \implies T(n) = O(n \log n)
    $$
\end{frame}


\begin{frame}
    \frametitle{References}
    \scriptsize
    \textbf{Cooley \& Tukey}: \textit{An algorithm for the machine calculation of complex Fourier series}

    \textbf{Johnson \& Frigo}: \textit{Implementing FFTs in practice}

    \textbf{Falch}: \textit{The Discrete and Fast Fourier Transforms}

    \textbf{Bogart \& Stein}: \textit{Discrete Math in Computer Science} (Section 5.2: The Master theorem)
\end{frame}


\end{document}
