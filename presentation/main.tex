\documentclass[14pt]{beamer}
\usepackage{mathtools}
\usepackage{emoji}
\usepackage{xcolor}
\usepackage{xfrac}
\usepackage{csquotes}

% This lets us \includestandalone tikz pictures in standalone documents.
\usepackage{tikz}
\usepackage[mode=buildnew]{standalone}

\makeatother
\renewcommand{\thefootnote}{
    \ifcase\value{footnote}
        \or*
        \or**
        \or***
        \or****
    \fi}
\makeatletter

\usefonttheme{serif}
\setbeamertemplate{navigation symbols}{}%remove navigation symbols

\setbeamerfont{title}{shape=\sc}
\setbeamerfont{frametitle}{shape=\sc}

\setbeamercolor{title}{fg=black}
\setbeamercolor{frametitle}{fg=black}

\newcommand{\Seq}[1]{\left\{ {#1} \right\}}
\newcommand{\Exp}[1]{e^{#1}}

\title{The Fourier transform and the FFT algorithm}
\author{Steffen Haug}
\date{}

\begin{document}
\begin{frame}
\titlepage
\end{frame}

\begin{frame}
    \centering
    \textsc{The Fourier transform}\\[1em]
    \begin{equation}
        \hat f (\xi) = \int\displaylimits_{\mathbb R} f(t) \Exp{-2 \pi i \xi t } dt
    \end{equation}\\[1em]
    % The fourier transform: Incredibly important for engineering applications
    % - The fourier transform f_hat gives for a frequency xi the "amount" of xi in the signal
    % - Unfeasible to evaluate automatically analytically

    \pause
    Analytic evaluation generally not feasible.
\end{frame}

\begin{frame}
    \begin{equation*}
        \hat f (\xi) = \int\displaylimits_{\mathbb R} f(t) \Exp{-2 \pi i \xi t } dt
    \end{equation*}\\[1em]

    Most engineering applications use its discrete counterpart -- the DFT:
    \begin{equation}
    X_k = \sum_{n=0}^{N-1} x_n \Exp{-\frac {2 \pi i k n} N}
    \label{eq:dft}
    \end{equation}
    % The sequence X is a sort of discrete counterpart to f_hat
    % Imagine x samples f
\end{frame}

\begin{frame}
    \begin{equation*}
    X_k = \sum_{n=0}^{N-1} x_n \Exp{-\frac {2 \pi i k n} N}
    \label{eq:dft}
    \end{equation*}\\[1em]
    \centering
    We want a way to compute this \textit{efficiently}.
\end{frame}



\begin{frame}
    \centering
    \textsc{\large Preliminaries}\\
    \textit{\scriptsize Eulers formula and the roots of unity}
\end{frame}


% eulers identiry: special case of eulers formula exp(ix) = cos = isin

\begin{frame}
    \centering
    \textsc{Eulers identity}\\[1em]

    \begin{equation}
        e^{i \pi} = -1
    \end{equation}
\end{frame}

\begin{frame}
    \begin{equation*}
        e^{i \pi} = -1
    \end{equation*}\\[1em]

    \centering
    Just a special case of $e^{i \vartheta}$ when $\vartheta = \pi$.
    \begin{equation}
        e^{i\vartheta} = \cos{\vartheta} + i \sin{\vartheta}
    \end{equation}
\end{frame}


\begin{frame}
    \begin{equation*}
        e^{i\vartheta} = \cos{\vartheta} + i \sin{\vartheta}
    \end{equation*}
    \begin{figure}
        \centering
        \includestandalone[width=.625\textwidth]{euler}
    \end{figure}
\end{frame}

\begin{frame}
    \begin{equation*}
        e^{i\pi} = -1
    \end{equation*}
    \begin{figure}
        \centering
        \includestandalone[width=.625\textwidth]{euler_pi}
    \end{figure}
\end{frame}

% roots of unity: exp(in/N) solutions to x^N = 1
% remember complex multiplication: multiply length (=1) and add angles. Arg xy = Arg x + Arg y
% symmetry: N even => each root has a opposite partner with twice the angle which just has a negative sign

\begin{frame}
    \centering
    \textsc{The roots of unity}\\[1em]
    \begin{equation*}
        z^N = 1
    \end{equation*}
\end{frame}

\begin{frame}
    \begin{equation*}
        z^N = 1
    \end{equation*}

    \begin{figure}
        \centering
        \includestandalone[width=.5\textwidth]{roots}
    \end{figure}
    \centering

    \textsc{Solutions}: $e^\frac {2\pi i n}{N}$ for $n$ in $0, \dots, N-1$.
\end{frame}

% \begin{frame}
%     \centering
%     \textsc{To verify}\\[1em]
%     \begin{align*}
%         \left(e^\frac{2 \pi i n} N\right)^N &=
%             \left(e^\frac{2 \pi i N} N\right)^n \\
%             &= \left(e^{2 \pi i}\right)^n\\
%             &= 1^n \\
%             &= 1
%     \end{align*}
% \end{frame}

\begin{frame}
    \centering
    \textsc{Important point}
    \begin{figure}
        \centering
        \includestandalone[width=.5\textwidth]{roots_mirror}
    \end{figure}
    When N is even, every root has a ``friend'' with the opposite sign.
\end{frame}


\begin{frame}
    \centering
    \textsc{\large The key idea}\\
    \textit{\scriptsize Decomposing the DFT}
\end{frame}

\begin{frame}
    \begin{center}
        \textsc{Simplifying assumption}\\[.5em]
        $N$ is a power of two\\[1.5em]
    \end{center}
\end{frame}

\begin{frame}
    Let $W = e^\frac{-2 \pi i} N$ (notation from Cooley-Tukey).

    We can restate \eqref{eq:dft} as
    \begin{equation}
        X_k = \sum_n x_n W^{nk}
    \end{equation}
\end{frame}

\begin{frame}
    \centering
    \textsc{\large Gaining speed}\\
    \textit{\scriptsize Reusing intermediate results}
\end{frame}

\begin{frame}
    \frametitle{References}

    \begin{itemize}
        \item \textbf{Cooley \& Tukey}: .....
    \end{itemize}
\end{frame}


\end{document}
